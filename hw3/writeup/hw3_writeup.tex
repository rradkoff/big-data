\documentclass[11pt, fleqn]{article}

\usepackage{amsmath}
\usepackage{amsfonts}
\usepackage[margin=1in]{geometry} % To set the margin widths
\usepackage{graphicx}
\usepackage{listings}
\usepackage{multirow}
\usepackage{tabularx}
\usepackage{varioref}
\usepackage{cleveref}
\usepackage{siunitx}
\usepackage{subcaption}
\usepackage{titlesec}
\usepackage{bm}

\lstset{
  language=R,
  literate = {<-}{{$\gets$}}1 {~}{{$\sim$}}1
}

\sisetup{output-exponent-marker=\textsc{e}}

\titleformat{\section}[block]{\bfseries}{\thesection}{1em}{}

\setlength{\parskip}{12pt} % Sets a blank line in between paragraphs
\setlength\parindent{0pt} % Sets the indent for each paragraph to zero

\begin{document}

\title{Big Data: Homework 3}
\author{Will Clark \& Matthew DeLio \\ 41201-01}
\date{\today}
\maketitle

\section{Player Contribution Regression}

The model for player contribution is:
\[ \log\left[\frac{Pr(y=1)}{1-Pr(y=1)}\right] = \beta_0 + \alpha_{\texttt{team,season}} + \alpha_{\texttt{config}} + \sum_{\texttt{homeplyr}} \beta_j + \sum_{\texttt{awayplyr}} \beta_j \]
For a given player \textit{j}, the model estimates the odds multiplier that a goal scored while player \textit{j} is on the ice was scored by the his team. It includes the following two control factors:
\begin{itemize}
  \item \texttt{team,season}: This should control for high- or low-offense years or certain arenas that provide a special home-ice advantage; and
  \item \texttt{config}: This should control for disproportionate playing time in power plays or end-of-game situations where the goalie has been pulled.
\end{itemize}
To use one example, the coefficient on Alex Ovechkin is 0.30. This means that a goal scored while Ovechkin is on the ice is $\exp(0.30)=1.35$ times as likely to be scored by his team, the Washington Capitols, than by their opponents.

We can sort the array of player coefficients to determine the 10 most and least valuable players in the data set. We show the results in Table~\ref{tab:10best} and Table~\ref{tab:10worst}. This evaluation metric accords with our intution about hockey. The list of 10 best players includes some of the sport's all-time greats, which tells us the model is doing a reasonably good job of quantifying performance.

By this performance metric, the best and worst players are both outliers. We can see in Figure~\ref{fig:play_rtg} that most players are centered around a 0 rating, and only a small handful of players are significantly better and significanlty worse than the mean.

\begin{table}[ht]
  \centering
  \begin{tabular}{l l l l}
    \hline
    Player & Rank & $\beta_j$ & $\exp(\beta_j)$ \\ 
    \hline
    Peter Forsberg &    1    &    0.7548 &    2.1272 \\  
    Tyler Toffoli  &    2    &    0.6293 &    1.8762 \\  
    Ondrej Palat   &    3    &    0.6284 &    1.8746 \\  
    Zigmund Palffy &    4    &    0.4427 &    1.5569 \\  
    Sidney Crosby  &    5    &    0.4131 &    1.5115 \\  
    Joe Thornton   &    6    &    0.3838 &    1.4678 \\  
    Pavel Datsyuk  &    7    &    0.3762 &    1.4567 \\  
    Logan Couture  &    8    &    0.3682 &    1.4451 \\  
    Eric Fehr      &    9    &    0.3677 &    1.4444 \\  
    Martin Gelinas &    10   &    0.3578 &    1.4301 \\   
    \hline
  \end{tabular}
\caption{Top 10 NHL Players (2002-2014)} 
\label{tab:10best}
\end{table}

\begin{table}[ht]
  \centering
  \begin{tabular}{l l l l}
    \hline
    Player & Rank & $\beta_j$ & $\exp(\beta_j)$ \\ 
    \hline
    Ryan Hollweg   &    2430 &    -0.2989 &    0.7417 \\  
    Raitis Ivanans &    2431 &    -0.3129 &    0.7313 \\  
    Darroll Powe   &    2432 &    -0.3340 &    0.7161 \\  
    Chris Dingman  &    2433 &    -0.3342 &    0.7159 \\  
    Mathieu Biron  &    2434 &    -0.3512 &    0.7038 \\  
    Thomas Pock    &    2435 &    -0.3844 &    0.6809 \\  
    Niclas Havelid &    2436 &    -0.3855 &    0.6801 \\  
    P.J. Axelsson &    2437 &    -0.4284 &    0.6516 \\  
    John McCarthy  &    2438 &    -0.5652 &    0.5683 \\  
    Tim Taylor     &    2439 &    -0.8643 &    0.4213 \\   
    \hline
  \end{tabular}
\caption{Bottom 10 NHL Players (2002-2014)} 
\label{tab:10worst}
\end{table}


\end{document}
